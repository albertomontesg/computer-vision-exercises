\documentclass{ethz_report}
\usepackage{listings}
\usepackage{color}
\usepackage{caption}
\usepackage{subcaption}


\definecolor{codegreen}{rgb}{0,0.6,0}
\definecolor{codegray}{rgb}{0.5,0.5,0.5}
\definecolor{codepurple}{rgb}{0.58,0,0.82}
\definecolor{backcolour}{rgb}{1,1,1}

\lstdefinestyle{mystyle}{
    backgroundcolor=\color{backcolour},
    commentstyle=\color{codegreen},
    keywordstyle=\color{magenta},
    numberstyle=\tiny\color{codegray},
    stringstyle=\color{codepurple},
    basicstyle=\ttfamily,
    breakatwhitespace=false,
    breaklines=true,
    captionpos=b,
    keepspaces=true,
    numbers=left,
    numbersep=5pt,
    showspaces=false,
    showstringspaces=false,
    showtabs=false,
    tabsize=4,
    frame=lines
}
\lstset{style=mystyle}

\title{Exercise 10 - Image Categorization}
\subject{Computer Vision}
\author{Alberto Montes}
\email{malberto@student.ethz.ch}
\date{\today}

\begin{document}
\maketitle

\section*{Local Feature Extraction}

To extract the local features for each image, first it requires to implement the function to extract the grid points positions for each image.
The implementation is on Listing~\ref{lst:grid_points} where the coordinates of the grid points are found and returned.

\lstinputlisting[language=MATLAB, caption=\texttt{grid\_points.m}, label={lst:grid_points}]{../code/grid_points.m}

The next step is for each of the points of the grid at each image, compute the Histogram of oriented Gradients for $4 \times 4$ grid pixels. The implementation is in Listing~\ref{lst:descriptors_hog} where for each points the descriptor is computed and also the patch of each descriptor is returned.

\lstinputlisting[language=MATLAB, caption=\texttt{descriptors\_hog.m}, label={lst:descriptors_hog}]{../code/descriptors_hog.m}

\section*{Codebook Construction}

Once with the descriptors and patches extracted for each image, is time to construct the codebook.
To do so, a k-means cluster algorithm is run to obtain an specific number of clusters centers to build the codebook. For the implementation, the codebook size its fixed in 200 codes.

For the k-means algorithm implementation first requires to implement the \texttt{findnn} (Listing~\ref{lst:findnn}) function which find to each descriptor the other ones which are closest.
With this function, the k-means algorithm has been implemented as shown in Listing~\ref{lst:kmeans}.

\lstinputlisting[language=MATLAB, caption=\texttt{findnn.m}, label={lst:findnn}]{../code/findnn.m}

\lstinputlisting[language=MATLAB, caption=\texttt{kmeans.m}, label={lst:kmeans}]{../code/kmeans.m}

Finally the whole pipeline of creating the codebook is coded in Listing~\ref{lst:create_codebook}.
The pipeline iterates over all the images and for each on first convert it to gray scale, then obtain the grid points coordinates with the \texttt{grid\_points} function. Then obtain the descriptor for each of this points and the image patch using the \texttt{descriptors\_hog} function.
Finally with all the image's descriptors, cluster them with the \texttt{kmeans} algorithm and obtain the cluster centers as classification codebook. The whole pipeline implementation is on Listing~\ref{lst:create_codebook}.

\lstinputlisting[language=MATLAB, caption=\texttt{create\_codebook.m}, label={lst:create_codebook}]{../code/create_codebook.m}

For the training data given with the assignment, and computing a codebook with size 200, the visualization of the codebook is on Figure~\ref{fig:visualize_codebook} where the patches of the closest descriptors to the codebook cluster's centers are plot.

\begin{figure}[h]
    \centering
    \includegraphics[width=1\linewidth]{images/visualize_codebook}
    \caption{Visualization of the codebook for $K=200$.}
    \label{fig:visualize_codebook}
\end{figure}

\section*{Bag-of-Words Image Representation}

Once the codebook has been defined, for each image is extracted a bunch of descriptors which each one will be match to one of the codebook's code or visual word.
The purpose of the bag-of-words is to represent an image as an histogram of the visual words appearing on it, and with this histogram then perform a classification.
To do so, first the \texttt{bow\_histogram} function has been implemented (in Listing~\ref{lst:bow_histogram}) which for each image given extract the histogram of visual words, given the codebook previously found.

\lstinputlisting[language=MATLAB, caption=\texttt{bow\_histogram.m}, label={lst:bow_histogram}]{../code/bow_histogram.m}

To perform a classification, is necessary to precompute this BoW histograms to all the training examples and store to further classify the test dataset. This is performed in the function \texttt{create\_bow\_histogram} on Listing~\ref{lst:create_bow_histograms}.

\lstinputlisting[language=MATLAB, caption=\texttt{create\_bow\_histograms.m}, label={lst:create_bow_histograms}]{../code/create_bow_histograms.m}

\section*{Nearest Neighbor Classification}


\lstinputlisting[language=MATLAB, caption=\texttt{bow\_recognition\_nearest.m}, label={lst:bow_recognition_nearest}]{../code/bow_recognition_nearest.m}

The result of the implementation with the default value of codebook size is a $76.76\%$ accuracy.

\section*{Bayesian Classification}

\lstinputlisting[language=MATLAB, caption=\texttt{bow\_recognition\_bayes.m}, label={lst:bow_recognition_bayes}]{../code/bow_recognition_bayes.m}

The result of the implementation with the default value of codebook size is a $70.70\%$ accuracy.

\section*{Results and Conclusions}

Create a table that for different values of K (sizeCodebook) the result for each of the classification methods.

% Size Codebook & Nearest Neighbor & Bayesian \\
%
% 10 & $50.50\%$ & $50.50\%$ \\
% 20 & $93.94\%$ & $98.99\%$ \\
% 50 & $96.97\%$ & $90.90\%$ \\
% 100 & $86.86\%$ & $83.83\%$ \\
% 150 & $81.81\%$ & $81.81\%$ \\
% 200 & $76.76\%$ & $70.70\%$ \\
% 250 & $64.64\%$ & $66.66\%$ \\
% 300 & $65.65\%$ & $62.62\%$ \\



\begin{figure}[h]
    \centering
    \begin{subfigure}[b]{.5\textwidth}
        \centering
        \includegraphics[width=1\linewidth]{images/visualize_codebook_20}
        \subcaption{$K=20$}
    \end{subfigure}%
    \begin{subfigure}[b]{.5\textwidth}
        \centering
        \includegraphics[width=1\linewidth]{images/visualize_codebook_50}
        \subcaption{$K=50$}
    \end{subfigure}
    \caption{Visualize codebook with better classification accuracy.}
    \label{fig:visualize_codebook_best}
\end{figure}

\end{document}
